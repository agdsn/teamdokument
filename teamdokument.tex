\input{preamble}
\begin{document}

\title{Grundsatzbeschluss mit Referenzcharakter zum Einsatz von Teams in der AG DSN}
\author{Arbeitsgemeinschaft Dresdner Studentennetz}
\date{Vom \today}
\maketitle

\begin{contract}

\Paragraph{title=Anwendungsbereich}

Dieser Beschluss schafft die Möglichkeit, Teams als Organe der Arbeitsgemeinschaft Dresdner Studentennetz (nachfolgend AG DSN) einzusetzen.

Es gelten die Regelungen der folgenden Ordnungen und Dokumente:
\begin{enumerate}
  \item Satzung der Arbeitsgemeinschaft Dresdner Studentennetz (Satzung AG DSN)
  \item Rahmennetzordnung für die Rechen- und Kommunikationstechnik und die Informationssicherheit an der TU Dresden (IuK-Rahmenordnung)
  \item Benutzungsordnung des Deutschen Forschungsnetzes (DFN)
  \item Rahmennetzordnung der AG DSN
  \item Rahmenvereinbarung mit dem Studentenwerk Dresden
  \item Anlage der Rahmenvereinbarung mit dem Studentenwerk Dresden
\end{enumerate}

Auf Grundlage dieses Beschlusses können Entscheidungen innerhalb der AG DSN nach Themen aufgeteilt werden und an Teams delegiert werden.

\Paragraph{title=Zuständigkeitsbereich}

Alle Teams der AG DSN sind Organe der AG DSN. Die AG DSN ist eine Arbeitsgemeinschaft des Studentenrates der Technischen Universität Dresden.

Die Zuständigkeit der Teams begrenzt sich durch eine thematische und eine geographische Komponente.

Geographische Komponente (betroffende Gebäude) sind
\begin{enumerate}
  \item die in der Anlage zur Rahmenvereinbarung mit dem Studentenwerk aufgeführten Wohnheime
  \item von der AG DSN angemietete Räumlichkeiten
\end{enumerate}

Die thematische Einschränkung eines Teams erfolgt über einen gesonderten Beschluss, welcher in §3 erläutert wird.
\Paragraph{title=Einrichtung und Auflösung}

Die Vollersammlung der AG DSN ist mit absoluter Mehrheit in der Lage einen Beschluss zu fassen, welcher Teams einrichtet oder auflöst.

Im Beschluss zur Einrichtung eines Teams muss die thematische Einschränkung des Teams klar ersichtlich sein.

Ein Beschluss nach diesem Paragraphen muss ein Teil der Tagesordnung der Vollversammlung sein.

Ein Beschluss zur Schaffung eines Teams muss in Schriftform vorliegen. Die Zweckbindung des Teams muss in dem Schriftstück dargelegt werden.

Die initialen Teammitglieder werden im Beschluss aufgelistet.

Ein Team muss über mindestens drei Mitglieder verfügen.

\Paragraph{title=Mitgliedschaft}

Ein Team kann durch einfachen Beschluss Personen, welche bereits Mitglieder der AG DSN sind, aufnehmen. Die Mitgliedschaft in der AG DSN ist durch die Satzung der AG DSN geregelt.

Die Mitgliedschaft in einem Team endet durch Rücktritt des Mitglieds oder das Ende der Mitgliedschaft in der AG DSN.

Ein Mitglied kann durch einen qualifizierten Beschluss durch ein Team aus diesem Team ausgeschlossen werden.

Eine Aufnahme oder ein Austritt aus einem Team der AG DSN berührt nicht die Mitgliedschaft in der AG DSN.

\Paragraph{title=Teamsitzung}

Jedes Team hält Teamsitzungen ab. Diese bestehen aus allen Mitgliedern des Teams. Diese sind stimmberechtigt.

In jedem Kalenderjahr muss mindestens eine Teamsitzung abgehalten werden.

Die Teamsitzungen fassen Beschlüsse für die gesamte AG DSN, welche auf die thematische Zweckbindung des jeweiligen Teams nach §2 beschränkt sind.

Die Teamsitzungen sind beschlussfähig, wenn mindestens die Hälfte der Mitglieder des jeweiligen Teams, jedoch wenigstens drei, anwesend sind.

Die Beschlüsse der Teams sind zu protokollieren und allen aktiven und beratenden Mitgliedern der AG DSN in geeigneter Weise zugänglich zu machen.

Eine Teamsitzung fasst einen
\begin{enumerate}
  \item einfachen Beschluss durch Zustimmung der Mehrheit,
  \item absoluten Beschluss durch Zustimmung der Hälfte oder
  \item qualifizierten Beschluss durch Zustimmung von Zweidrittel
\end{enumerate}
der abgegeben Stimmen`. Sofern nicht anders geregelt, ist ein einfacher Beschluss ausreichend.

Für folgende Beschlüsse ist eine Ankündigung von einer Woche vor der jeweiligen Teamsitzung nötig:
\begin{enumerate}
  \item Aufnahme eines Mitglieds der AG DSN in das Team
  \item Ausschluss eines Teammitglieds aus dem Team
  \item Wahl des Teamsprechers und seines Stellvertreters
\end{enumerate}

\Paragraph{title=Teamsprecher}

Jedes Team wählt einen Teamsprecher und einen stellvertretenden Teamsprecher, um die Kommunikation innerhalb der AG DSN sicher zu stellen.

Der Teamsprecher erfüllt folgende Aufgaben:
\begin{enumerate}
  \item Berichterstattung an den Vorstand der AG DSN
  \item Sicherstellung der Zugänglichkeit aller Protokolle des Teams für alle aktiven und beratenden Mitglieder der AG DSN
  \item Berichterstattung der Arbeit des Teams gegenüber der Vollversammlung der AG DSN
  \item Organisation der Teamtreffen
\end{enumerate}

Weitere Befugnisse des Teamsprechers können durch Beschluss der Teamsitzung gewährt werden.

Der Teamsprecher und sein Stellvertreter müssen einmal im Jahr durch absoluten Beschluss gewählt werden`. Falls im ersten Wahlgang keine absolute Mehrheit erreicht werden konnte, wird eine Stichwahl zwischen den beiden Kandidaten mit den meisten Stimmen durchgeführt.

Zur Berichterstattung an den Vorstand nimmt jeder Teamsprecher an den Vorstandssitzungen der AG DSN teil.

Die Aufgaben des Teamsprechers werden in seiner Abwesenheit vom stellvertretenden Teamsprecher übernommen.

\Paragraph{title=Finanzierung von Teams}

Teams können Budgets zur Finanzierung von Projekten in der Vollversammlung der AG DSN beantragen.

Beantragte Budgets müssen mit einfacher Mehrheit genehmigt werden.

Budgets für Teams dürfen nur im direkten Zusammenhang für den jeweiligen Zweck des Budgets eingesetzt werden.

Budgets sind auf einzelne abgegrenzte Zwecke (Zweckbindung), Zeiträume (Verfügbarkeit) und Geldmittel (Volumen) beschränkt. Durch erneute Abstimmung können Zeiträume für Budgets verlängert oder verkürzt werden.

Sektionen der AG DSN oder der Vorstand der AG DSN können weitere Budgets für Teams zur Verfügung stellen. Diese müssen ebenfalls über klar definierte Zweckbindung, Verfügbarkeit und Volumen verfügen.

Geldmittel, welche für die Deckung der Budgets benötigt werden, werden nach Mitgliederzahlen gleichmäßig aufgeteilt mit dem Sektionsbeitrag von den Konten der Sektionen der AG DSN auf das Konto des Vorstands der AG DSN übertragen.

Die Überweisungen der Teams werden vom Konto des Vorstandes der AG DSN bezahlt. Sie werden durch die Kontoberechtigten der AG DSN getätigt.

Eine Übersicht laufender und geplanter Budgets wird vom Schatzmeister der AG DSN erstellt und laufend aktuell gehalten. Diese Übersicht steht allen aktiven und beratenden Mitgliedern der AG DSN ständig zur Verfügung.

\Paragraph{title=Salvatorische Klausel}
Sollten einzelne Bestimmungen dieser Formulierungen unwirksam sein oder nach deren Beschluss unwirksam werden, so ist dadurch die Wirksamkeit der anderen Formulierungen nicht berührt`. An die Stelle dieser unwirksamen Formulierungen treten diejenigen Gesetze und Regelungen, welche der ursprünglichen Intention am meisten entsprechen.

\Paragraph{title=Inkrafttreten}
Dieser Grundsatzbeschluss tritt am \formatdate{1}{8}{2015} in Kraft.
\end{contract}
\end{document}

