\documentclass[draft,parskip=half-,DIV=12,mpinclude]{scrartcl}
\usepackage[utf8]{inputenc}
\usepackage[T1]{fontenc}
\usepackage[ngerman]{babel}
\useshorthands{'}
\defineshorthand{'S}{\Sentence\ignorespaces}
\defineshorthand{'.}{. \Sentence\ignorespaces}
\usepackage{tgheros}
\usepackage{tgpagella}
\usepackage{tgcursor}
\usepackage{datetime}
\usepackage{changes}
\usepackage{scrjura}

\AfterCalculatingTypearea{%
  \addtolength{\marginparwidth}{\marginparwidth}%
}
\recalctypearea
\setremarkmarkup{\marginpar{\scriptsize#1:\\#2}}
\setaddedmarkup{\textcolor{green}{#1}}
\setdeletedmarkup{\textcolor{red}{\sout{#1}}}

\definechangesauthor[name={Sebastian Schrader}]{shr}

\definechangesauthor[name={Friedrich Zahn}]{tst}

\begin{document}

\title{Grundsatzbeschluss mit Referenzcharakter zum Einsatz von Teams in der AG DSN}
\author{Arbeitsgemeinschaft Dresdner Studentennetz}
\date{Vom \today}
\maketitle

\begin{contract}

\Clause{title=Anwendungsbereich}

'S Dieser \replaced[id=shr]{Grundsatzbeschluss}{Beschluss} schafft die Möglichkeit, Teams als Organe der Arbeitsgemeinschaft Dresdner Studentennetz (nachfolgend AG DSN) einzusetzen.

'S \deleted[id=shr,remark=Verweis auf übergeordnete Ordnungen ist nicht notwendig. Diese gelten sowieso.]{Es gelten die Regelungen der folgenden Ordnungen und Dokumente:}
\begin{enumerate}
  \item \deleted{Satzung der Arbeitsgemeinschaft Dresdner Studentennetz (Satzung AG DSN)}
  \item \deleted{Rahmennetzordnung für die Rechen- und Kommunikationstechnik und die Informationssicherheit an der TU Dresden (IuK-Rahmenordnung)}
  \item \deleted{Benutzungsordnung des Deutschen Forschungsnetzes (DFN)}
  \item \deleted{Rahmennetzordnung der AG DSN}
  \item \deleted{Rahmenvereinbarung mit dem Studentenwerk Dresden}
  \item \deleted{Anlage der Rahmenvereinbarung mit dem Studentenwerk Dresden}
\end{enumerate}

'S Auf Grundlage dieses Beschlusses können Entscheidungen innerhalb der AG DSN nach Themen aufgeteilt werden und an Teams delegiert werden.

\SubClause{title={\added[id=shr,remark={Dies wurde ursprünglich nur für die Team-Sitzung geregelt, ist nun aber ein eigener, unabhängiger Paragraph}]{Beschlüsse}}}

'S \added[id=shr]{Im Sinne dieses Dokuments wird ein}
\begin{enumerate}
  \item \added[id=shr]{einfacher Beschluss durch Zustimmung der Mehrheit},
  \item \added[id=shr]{absoluter Beschluss durch Zustimmung der Hälfte, oder}
  \item \added[id=shr]{qualifizierter Beschluss durch Zustimmung von Zweidrittel}
\end{enumerate}
\added[id=shr]{der abgegeben Stimmen gefasst}.
'S \added[id=shr]{Enthaltungen gelten als nicht abgegebene Stimme.}
'S \added[id=shr]{Sofern nicht anders geregelt, ist ein einfacher Beschluss ausreichend.}

\SubClause{title={\added[id=shr,remark={Die Vollversammlung unserer aktuelle Satzung als Basis zu nehmen ist problematisch, daher schlage ich die Team-Vollverersammlung vor.}]{Team-Vollversammlung}}}

'S \added[id=shr]{
  Die Team-Vollversammlung ist ein Organ der AG DSN und besteht aus allen Mitgliedern der AG DSN.
  'S Stimmberechtigt sind alle aktiven und beratenden Mitglieder.
}

'S \added[id=shr]{
  Die Team-Vollversammlung ist beschlussfähig wenn mindestens die Hälfte der aktiven und beratenden Mitglieder anwesend ist.
}

'S \added[id=shr]{
  'Die Team-Vollversammlung wird vom Vorstand der AG DSN einberufen.
}

'S  \added[id=shr]{Die Bekanntgabe des Termines und der Tagesordnung muss mindestens 14 Tage im Voraus erfolgen.}

'S \added[id=shr]{Die Team-Vollversammlung entscheidet über}
\begin{enumerate}
  \item \added[id=shr]{Die Einrichtung und Auflösung von Teams}
  \item \added[id=shr]{Budget-Anträge von Teams}
  \item \added[id=shr]{Kompetenz-Streitigkeiten zwischen Teams}
  \item \added[id=shr]{Aufnahme und Auschluss von Team-Mitgliedern}
\end{enumerate}

\Clause{title=Zuständigkeitsbereich}

'S \replaced[id=shr,remark={Vereinfachung}]{Teams sind Organe der AG DSN.}{Alle Teams der AG DSN sind Organe der AG DSN.}
'S \deleted[id=shr]{Die AG DSN ist eine Arbeitsgemeinschaft des Studentenrates der Technischen Universität Dresden.}

'S \replaced[id=shr,remark={Eine geographische Begrenzung ergibt sich bereits aus unseren Vereinbarung mit dem Studentenwerk, ZIH, usw.}]{%
  Die Zuständigkeit der Teams ist begrenzt.}{%
  Die Zuständigkeit der Teams begrenzt sich durch eine thematische und eine geographische Komponente}.
'S \added[id=shr]{Sie muss von der Team-Vollversammlung bei der Einrichtung eines Teams schriftlich festgelegt werden und kann durch Beschluss der Team-Vollversammlung angepasst werden.}

'S \deleted[id=shr]{Geographische Komponente (betroffende Gebäude) sind}
\begin{enumerate}
  \item \deleted[id=shr]{die in der Anlage zur Rahmenvereinbarung mit dem Studentenwerk aufgeführten Wohnheime}
  \item \deleted[id=shr]{von der AG DSN angemietete Räumlichkeiten}
\end{enumerate}

'S \deleted[id=shr]{Die thematische Einschränkung eines Teams erfolgt über einen gesonderten Beschluss, welcher in §3 erläutert wird.}

\Clause{title=Einrichtung und Auflösung}

'S \replaced[id=shr]{%
  Die Team-Vollversammlung der AG DSN kann mittels absoluten Beschlusses Teams einrichten oder auflösen.%
}{%
  Die Vollverersammlung der AG DSN ist mit absoluter Mehrheit in der Lage einen Beschluss zu fassen, welcher Teams einrichtet oder auflöst.%
}

\deleted[id=shr,remark=redundant]{'S Im Beschluss zur Einrichtung eines Teams muss die thematische Einschränkung des Teams klar ersichtlich sein.}

'S \replaced[id=shr]{%
  Die Einrichtung oder Auflösung eines Teams muss als Tagesordnungspunkt in der Einladung zur Versammlung enthalten sein.%
}{%
  Ein Beschluss nach diesem Paragraphen muss ein Teil der Tagesordnung der Vollversammlung sein.%
}

\deleted[id=shr,remark=redundant]{%
  'S Ein Beschluss zur Schaffung eines Teams muss in Schriftform vorliegen.%
  'S Die Zweckbindung des Teams muss in dem Schriftstück dargelegt werden.%
}

'S Die initialen Teammitglieder werden im Beschluss aufgelistet.

'S Ein Team muss \added[id=shr,remark={Diese Anforderung sollte nur bei der Einrichtung bestehen, denn was sollte sonst passieren? Würde das Team dann aufgelöst? Teams mit weniger als 3 Mitglieder sind ohnehin nicht beschlussfähig.}]{initial} über mindestens drei Mitglieder verfügen.

\Clause{title=Mitgliedschaft}

'S \added[id=shr]{Vorraussetzung für die Mitgliedschaft in einem Team ist der Mitgliedsstatus als aktives oder beratendendes Mitglied der AG DSN.}

'S \replaced[id=shr,remark=]{Mitglieder werden durch einfachen Beschluss der Teamsitzung oder der Team-Vollversammlung in ein Team aufgenommen.}{Ein Team kann durch einfachen Beschluss Personen, welche bereits Mitglieder der AG DSN sind, aufnehmen.}
'S \deleted[id=shr]{Die Mitgliedschaft in der AG DSN ist durch die Satzung der AG DSN geregelt.}

'S \deleted[id=shr]{Die Mitgliedschaft in einem Team endet durch Rücktritt des Mitglieds oder das Ende der Mitgliedschaft in der AG DSN.}

'S \deleted[id=shr]{Ein Mitglied kann durch einen qualifizierten Beschluss durch ein Team aus diesem Team ausgeschlossen werden.}

'S \deleted[id=shr,remark=Warum sollte sie auch?]{Eine Aufnahme oder ein Austritt aus einem Team der AG DSN berührt nicht die Mitgliedschaft in der AG DSN.}

'S \added[id=shr,remark={Übersichtlichere Regelung des Endes der Mitgliedschaft}]{Die Mitgliedschaft in einem Team endet durch}
\begin{enumerate}
  \item \added[id=shr]{Erklärung des Mitglieds gegenüber dem Teamsprecher,}
  \item \added[id=shr]{Verlust der Vorraussetzung für die Mitgliedschaft in Teams, oder}
  \item \added[id=shr]{Auschluss mittels qualifizierten Beschlusses des Teams oder der Team-Vollversammlung.}
\end{enumerate}

\Clause{title=Teamsitzung}

'S Jedes Team hält Teamsitzungen ab.
'S Diese bestehen aus allen Mitgliedern des Teams.
'S Diese sind stimmberechtigt.

'S In jedem Kalenderjahr muss mindestens eine Teamsitzung abgehalten werden.

'S Die Teamsitzungen fassen Beschlüsse für die gesamte AG DSN, welche auf die thematische Zweckbindung des jeweiligen Teams nach §2 beschränkt sind.

'S \added[id=tst]{Ist ein Mitglied mehr als drei Teamsitzungen eines Teams nicht anwesend, wird gilt es als ruhend in diesem Team. Sobald es wieder eine Teamsitzung des Teams besucht, ist es nicht mehr ruhend.}

'S Die Teamsitzungen sind beschlussfähig, wenn mindestens die Hälfte der \added[id=tst]{nicht ruhenden} Mitglieder des jeweiligen Teams, jedoch wenigstens drei, anwesend sind.

'S \added[id=tst]{Änderungen der Mitgliederstruktur sind stets in der nächsten Teamsitzung durch den Teamsprecher bekanntzugeben und im Protokoll zu vermerken.}

'S \replaced[id=tst]{%
  Teamsitzungen sind in einem Ergebnisprotokoll festzuhalten.
  'S Dieses muss innerhalb einer Woche fertiggestellt und allen aktiven und beratenden Mitgliedern der AG DSN in geeigneter Weise zugänglich gemacht werden.%
}{%
  Die Beschlüsse der Teams sind zu protokollieren und allen aktiven und beratenden Mitgliedern der AG DSN in geeigneter Weise zugänglich zu machen.%
}

'S \deleted[id=shr,remark={Nun allgemein geregelt}]{Eine Teamsitzung fasst einen}
\begin{enumerate}
  \item \deleted[id=shr]{einfachen Beschluss durch Zustimmung der Mehrheit,}
  \item \deleted[id=shr]{absoluten Beschluss durch Zustimmung der Hälfte oder}
  \item \deleted[id=shr]{qualifizierten Beschluss durch Zustimmung von Zweidrittel}
\end{enumerate}
\deleted[id=shr]{der abgegeben Stimmen.
'S Sofern nicht anders geregelt, ist ein einfacher Beschluss ausreichend.}

'S Für folgende Beschlüsse ist eine Ankündigung von einer Woche vor der jeweiligen Teamsitzung nötig:
\begin{enumerate}
  \item \deleted[id=tst,remark=Aufnahmen sollten möglichst einfach sein]{Aufnahme eines Mitglieds der AG DSN in das Team}
  \item Ausschluss eines Teammitglieds aus dem Team
  \item Wahl des Teamsprechers und seines Stellvertreters
\end{enumerate}

\Clause{title=Teamsprecher}

'S Jedes Team wählt einen Teamsprecher und einen stellvertretenden Teamsprecher \added[id=tst]{aus den Mitgliedern des Teams}, um die Kommunikation innerhalb der AG DSN sicher zu stellen.

'S Der Teamsprecher erfüllt folgende Aufgaben:
\begin{enumerate}
  \item Berichterstattung an den Vorstand der AG DSN
  \item Sicherstellung der Zugänglichkeit aller Protokolle des Teams für alle aktiven und beratenden Mitglieder der AG DSN
  \item Berichterstattung der Arbeit des Teams gegenüber der \added[id=shr]{Team-}Vollversammlung der AG DSN
  \item Organisation der \replaced[id=shr]{Teamsitzungen}{Teamtreffen}
  \item \added[id=tst]{Führen der Mitgliedsliste des Teams in nachvollziehbarer Weise}
\end{enumerate}

'S Weitere Befugnisse des Teamsprechers können durch Beschluss der Teamsitzung gewährt werden.

'S Der Teamsprecher und sein Stellvertreter müssen einmal im Jahr durch absoluten Beschluss gewählt werden.
'S Falls im ersten Wahlgang keine absolute Mehrheit erreicht werden konnte, wird eine Stichwahl zwischen den beiden Kandidaten mit den meisten Stimmen durchgeführt.

'S Zur Berichterstattung an den Vorstand nimmt jeder Teamsprecher an den Vorstandssitzungen der AG DSN teil.

'S Die Aufgaben des Teamsprechers werden in seiner Abwesenheit vom stellvertretenden Teamsprecher übernommen.

\Clause{title=Finanzierung von Teams}

'S Teams können Budgets zur Finanzierung von Projekten in der \added[id=shr]{Team-}Vollversammlung der AG DSN beantragen.

'S Beantragte Budgets müssen mit einfacher Mehrheit genehmigt werden.

'S \added[id=shr]{Ein Budget-Antrag muss mindestens die folgenden Punkte enthalten:}
\begin{enumerate}
  \item \added[id=shr]{Eindeutige Identifikationsnummer}
  \item \added[id=shr]{Titel}
  \item \added[id=shr]{Zweck}
  \item \added[id=shr]{Betrag}
  \item \added[id=shr]{eventuelle Befristung}
  \item \added[id=shr]{Begründung}
\end{enumerate}


'S \replaced[id=shr]{%
  Wird einem Budget-Antrag durch Beschluss der Team-Vollversammlung statt gegeben, so sind die im Antrag festgelegten Einschräkungen bindend.%
}{%
  Budgets für Teams dürfen nur im direkten Zusammenhang für den jeweiligen Zweck des Budgets eingesetzt werden.%
}

'S \deleted[id=shr]{Budgets sind auf einzelne abgegrenzte Zwecke (Zweckbindung), Zeiträume (Verfügbarkeit) und Geldmittel (Volumen) beschränkt.}
\replaced[id=shr]{%
  Der Betrag, die Befristung oder der Zweck eines Budgets kann durch Beschluss der Team-Vollversammlung geändert werden.%
}{%
  'S Durch erneute Abstimmung können Zeiträume für Budgets verlängert oder verkürzt werden.%
}

'S Sektionen der AG DSN oder der Vorstand der AG DSN können weitere Budgets für Teams zur Verfügung stellen.
'S Diese müssen ebenfalls über klar definierte Zweckbindung, Verfügbarkeit und Volumen verfügen.

'S Geldmittel, welche für die Deckung der Budgets benötigt werden, werden nach Mitgliederzahlen gleichmäßig aufgeteilt mit dem Sektionsbeitrag von den Konten der Sektionen der AG DSN auf das Konto des Vorstands der AG DSN übertragen.

'S Die Überweisungen der Teams werden vom Konto des Vorstandes der AG DSN bezahlt.
'S Sie werden durch die Kontoberechtigten der AG DSN getätigt.

'S Eine Übersicht \replaced[id=tst,remark=besser?]{beantragter und bewilligter}{laufender und geplanter} Budgets wird vom Schatzmeister der AG DSN erstellt und laufend aktuell gehalten.
'S Diese Übersicht steht allen aktiven und beratenden Mitgliedern der AG DSN ständig zur Verfügung.

\Clause{title=Salvatorische Klausel}
'S Sollten einzelne Bestimmungen dieser Formulierungen unwirksam sein oder nach deren Beschluss unwirksam werden, so ist dadurch die Wirksamkeit der anderen Formulierungen nicht berührt.
'S An die Stelle dieser unwirksamen Formulierungen treten diejenigen Gesetze und Regelungen, welche der ursprünglichen Intention am meisten entsprechen.

\Clause{title=Inkrafttreten}
'S Dieser Grundsatzbeschluss tritt am \formatdate{1}{8}{2015} in Kraft.

\end{contract}
\end{document}

